\documentclass[10pt]{article}
\usepackage{latexsym}
\usepackage{amsmath}
\usepackage{natbib}
\usepackage{graphicx}
\usepackage{listings}
\usepackage{amssymb} 
\usepackage{enumerate}

\renewcommand{\P}{\mathbb{P}}
\newcommand{\Cov}{\mathrm{Cov}}
\newcommand{\E}{\mathrm{E}}
\newcommand{\Var}{\mathrm{Var}}

\setlength\parindent{0pt}

\title{MCMT Homework 16}
\author{Shun Zhang}
\date{}

\begin{document}
\maketitle

\section*{Exercise 16.1}

Assume G is transient. For a vertex $a$, there exists a flow from $a$ to
inifinity with finite energy. Consider its k-fuzz graph, we just set the
flow on the additional edges to be 0. Then the new flow is a finite flow for the
k-fuzz graph.

\hfill

If the k-fuzz of a graph is transient. Consider deriving the original graph $G$ by
removing edges. Consider an edge $e'$ in k-fuzz that connects two vertices with
distance of $k$.  Let the path in G that connects these two vertices be $e_1,
e_2, \cdots, e_k$.  By removing $e'$, let $\theta(e')$ go through $e_1, e_2,
\cdots, e_k$ instead.

All the edges have unit conductances. So the energy before removing $e'$ is
$\sum_{i=1}^k \theta(e_i)^2 + \theta(e')^2$. The energy after removing $e'$ is 
$\sum_{i=1}^k (\theta(e_i) + \theta(e'))^2\\
= \sum_{i=1}^k \theta(e_i)^2 + 2 \sum_{i=1}^k \theta(e_i) \theta(e') + k
\theta(e')^2\\
\leq \sum_{i=1}^k \theta(e_i)^2 + \sum_{i=1}^k (\theta(e_i)^2 + \theta(e')^2) +
k \theta(e')^2\\
\leq 2k (\sum_{i=1}^k \theta(e_i)^2 + \theta(e')^2)
$.

So the energy increment by removing one edge in k-fuzz graph can be locally
finitelly bounded.  For any original edge in G, its energy can be augemented at
most $(\Delta^{k+1})^2$ times, where $\Delta^{k+1}$ upper bounds the number of
the nodes in a tree with depth of $k$ and branching factor of $\Delta$. So the
energy of the flow can be increased at most $(2k)^{(\Delta^{k+1})^2}$ times
(which is a loose upper bound), which is still finite.

\end{document}
