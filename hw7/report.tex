\documentclass[10pt]{article}
\usepackage{latexsym}
\usepackage{amsmath}
\usepackage{natbib}
\usepackage{graphicx}
\usepackage{listings}
\usepackage{amssymb} 

\renewcommand{\P}{\mathbb{P}}

\title{MCMT Homework 7}
\author{Shun Zhang}
\date{}

\begin{document}
\maketitle

\section*{Exercise 7.1}

For any state $x$, 
$\sum_{y \in \Omega} \hat{P}(x, y)
= \dfrac{\sum_{y \in \Omega} \pi(y) P(y, x)}{\pi(x)}
= \dfrac{\pi(x)}{\pi(x)}
= 1
$.
We know $\pi, P$ are all non-negative. Multiplicaiton is closed for non-negative
numbers, so $\hat{P}$ is non-negative.

So $\hat{P}$ is a stochastic matrix.

\section*{Exercise 7.2}

Show $\hat{P}^n (n, \cdot) = \pi$ by induction on the number of states in the
Markov Chain, which is $n+1$ (from 0 to n).

Base case: when $n = 0$, the Markov Chain has 1 state. $\hat{P}^0 (0, \cdot) =
\delta_0$. This is the stationary distribution.

Inductive step: assume that $\hat{P}^n (n, \cdot) = \pi$ for some $n \geq 0$.
Consider a transition matrix $Q$ on a Markov chain with $n+2$ states.
 
$$Q^n(n+1, m) = \left\{
\begin{array}{l l}
\pi(m - 1) & m > 0 \\
0 & m = 0
\end{array}
\right.
$$
by induction hypothesis.

Consider $Q^{n+1}(n+1, \cdot)$, which is $Q^{n}(n+1, \cdot) Q$.

$Q^{n+1}(n+1, 0) = Q^{n}(n+1, 1) Q(1, 0) = \dfrac{1}{2} 1 = \dfrac{1}{2}$.

$Q^{n+1}(n+1, 1) = Q^{n}(n+1, 2) Q(2, 1) = \dfrac{1}{2^2} 1 = \dfrac{1}{2^2}$.

$\cdots$

$Q^{n+1}(n+1, n-1) = Q^{n}(n+1, n) Q(n, n - 1) = \dfrac{1}{2^n} 1 =
\dfrac{1}{2^{n}}$.

$Q^{n+1}(n+1, n) = Q^{n}(n+1, n+1) Q(n+1, n) = \dfrac{1}{2^n} \dfrac{1}{2} =
\dfrac{1}{2^{n+1}}$.

$Q^{n+1}(n+1, n+1) = Q^{n}(n+1, n+1) Q(n+1, n+1) = \dfrac{1}{2^{n}} \dfrac{1}{2}
= \dfrac{1}{2^{n+1}}$.

This is the stationary distribution for the Markov chain with $n+2$ states.

\end{document}
