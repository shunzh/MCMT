\documentclass[10pt]{article}
\usepackage{latexsym}
\usepackage{amsmath}
\usepackage{natbib}
\usepackage{graphicx}
\usepackage{listings}
\usepackage{amssymb} 

\renewcommand{\P}{\mathbb{P}}

\title{MCMT Homework 1}
\author{Shun Zhang}
\date{}

\begin{document}
\maketitle

\section*{Exercise 1.1}

Let $\mu(x) = \P(X_0 = x)$.

Base Case: The distribution of $X_0$ is characterized as $\mu$ as definition.

Inductive step: Assume that for some $n \geq 0$, $\P(X_n = x)$ can be
represented with $P$ and $\mu$. $\P(X_{n+1} = y) = \P(X_n = x) \P(X_{n+1} = y |
X_n = x) = \P(X_n = x) P(x, y)$. We already know that $\P(X_n = x)$ can be
represented with $P$ and $\mu$. So $\P(X_{n+1} = y)$ can also be represented
using $P$ and $\mu$.

\section*{Exercise 1.2}

$\P(X_1 = y)
= \sum_{x \in \Omega} \P(X_1 = y | X_0 = x) P(X_0 = x)
= \sum_{x \in \Omega} P(x, y) P(X_0 = x)
= \sum_{x \in \Omega} P_{xy} \mu_x
= \begin{pmatrix}
  \mu_1 & \mu_2 & \cdots & \mu_k
  \end{pmatrix}
  \begin{pmatrix}
  P_{1y} \\
  P_{2y} \\
  \cdots \\
  P_{ky}
  \end{pmatrix}
$.

The distribution of $X_1$ is $(\P(X_1 = 1), \cdots, \P(X_1 = k))
= \begin{pmatrix}
  \mu_1 & \mu_2 & \cdots & \mu_k
  \end{pmatrix}
  \begin{pmatrix}
  P_{11} & \cdots & P_{1k} \\
  P_{21} & \cdots & P_{2k} \\
  \cdots \\
  P_{ky} & \cdots & P_{kk} \\
  \end{pmatrix}
= \mu P
$.

To show the distribution of $X_t$ is $\mu P^t$ by way of induction. the base
case is shown above. Assume for some $n > 0$, the distribution of $X_n$ is  $\mu
P^n$, consider the distribution of $X_{n+1}$.

$\P(X_{n+1} = y) 
= \sum_{x \in \Omega} \P(X_{n+1} = y | X_n = x) P(X_n = x)
$. Similar to the reasoning in the base case, this is
$ \mu P^n
  \begin{pmatrix}
  P_{1y} \\
  P_{2y} \\
  \cdots \\
  P_{ky}
  \end{pmatrix}
$. So the distribution of $X_{n+1}$ is $\mu P^n P = \mu P^{n+1}$.

\end{document}
