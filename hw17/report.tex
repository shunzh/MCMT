\documentclass[10pt]{article}
\usepackage{latexsym}
\usepackage{amsmath}
\usepackage{natbib}
\usepackage{graphicx}
\usepackage{listings}
\usepackage{amssymb} 
\usepackage{enumerate}

\renewcommand{\P}{\mathbb{P}}
\newcommand{\Cov}{\mathrm{Cov}}
\newcommand{\E}{\mathrm{E}}
\newcommand{\Var}{\mathrm{Var}}

\setlength\parindent{0pt}

\title{MCMT Homework 17}
\author{Shun Zhang}
\date{}

\begin{document}
\maketitle

\section*{Exercise 17.1}

\begin{enumerate}
\item Consider $v, \lambda$ so that $Pv = \lambda v$.
Let $v_k$ be the component of $x$ with the maximum magnitude.
$\lambda |v_k| = |\sum_i P_{ik} v_i| \leq \sum_i P_{ik} |v_i| \leq \sum_i P_{ik}
|x_k| = |x_k|$.
So $\lambda \leq 1$.

\item It is proved that the nullity of $P - I$ is 1. $(1,1,\
\cdots,1)^T$ is a solution to $Pv = v$, so it is the unique solution.

\item Suppose that $Pv=-v$. Let $v=v^+ - v^-$ where $v^+$ and $v^-$ have
non-negative coordinates and disjoint support.

Because $P$ perserves the sum of components of $v$, so

$\sum P v^+ = \sum v^+\\
\sum P v^- = \sum v^-\\
\sum P v = \sum v \Rightarrow \sum P(v^+ - v^-) = \sum (v^- - v^+)
$.

By these linear equations, we have $\sum v^+ = \sum v^-$.

Because $P(v^+ - v^-) = v^- - v^+$, where $P v^+, P v^-, v^+, v^-$ are all
nonnegative vectors. $\sum P v^+ = \sum P v^- = \sum v^+ = \sum v^-$. We have $P
v^+ = v^-, P v^- = v^+$.

For a state $x$ in the support of $v^+$, we know $P(x, x) = 0$ because $Pv^+ =
v^-$. Because $P^{2k+1}v^+ = v^-$, $P^{2k + 1}(x, x) = 0$. As the transition
matrix is irreducible, the period must be a multiple of 2.

\end{enumerate}

\end{document}
