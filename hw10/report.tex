\documentclass[10pt]{article}
\usepackage{latexsym}
\usepackage{amsmath}
\usepackage{natbib}
\usepackage{graphicx}
\usepackage{listings}
\usepackage{amssymb} 

\renewcommand{\P}{\mathbb{P}}
\newcommand{\Cov}{\mathrm{Cov}}
\newcommand{\E}{\mathrm{E}}
\newcommand{\Var}{\mathrm{Var}}

\title{MCMT Homework 10}
\author{Shun Zhang}
\date{}

\begin{document}
\maketitle

\section*{Exercise 10.1}

$\mu((1, 2, \cdots, k) \rightarrow (2, 3, \cdots, k, 1)) = \frac{1}{n}$.

$\mu(\cdot) = 0$, for other permutations.

\section*{Exercise 10.2}

\begin{enumerate}
\item Consider when $\tau = 3$.

When $X_3 = X_0$, we may swap a card with itself at each time, and swap three
different cards for $t = 1, 2, 3$. There are $3 * 2 = 6$ ways. We can also swap
two cards back and forth for the first two times, and swap the third card with
itself. There are $3 * 2 = 6$ ways. So there are 12 ways to get $X_3 = X_0$.

When $X_3' = (2 1 3)X_0$, which we stop at swaping two cards on the top. We can
swap any card with itself in first two rounds, other than 1 and 2, in first two
rounds, and then swap 1 and 2 in the third time. We exclude the case of ``1
and 2'' in the first two rounds, because otherwise 3 won't be marked. There are
$5 * 2 = 10$ ways to do so.
We can also swap the third card with either of the top cards back and forth for
the first two times, and swap 1 and 2 in the third time. There are $2 * 2 * 2 =
8$ ways to do so. There are 18 ways for this case.

However, $\pi(X_3) = \pi(X_3')$ but $P(\tau = 3, X_3) \neq P(\tau = 3, X_3')$.
This is not a strong stationary time.

\item Consider when $\tau = 3$.
\end{enumerate}

\end{document}
