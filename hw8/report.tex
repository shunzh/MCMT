\documentclass[10pt]{article}
\usepackage{latexsym}
\usepackage{amsmath}
\usepackage{natbib}
\usepackage{graphicx}
\usepackage{listings}
\usepackage{amssymb} 

\renewcommand{\P}{\mathbb{P}}

\title{MCMT Homework 8}
\author{Shun Zhang}
\date{}

\begin{document}
\maketitle

\section*{Exercise 8.1}

For any non-negative constant $c$, define $\{\tau \leq t\} = 1_{|\{X_0, \cdots,
X_t\}| = c}$.  Then $\tau = c$ is a stopping time.

Suppose there are two stopping times $\tau_1$ and $\tau_2$.
We can make $\{X_0, \cdots, X_s\}$ satisfy $\{s \leq \tau_1\}$ and $\{X_{s},
\cdots, X_t\}$ satisfy $\{t - s \leq \tau_2\}$.
That is, $\{\tau_1 + \tau_2 \leq t\} = \{\tau_1 \leq s\} \land \{\tau_2 \leq t -
s, X_0 = y\}$, where $y$ is the state when $\{\tau_1 \leq s\}$ is true.
So $\tau_1 + \tau_2$ is a stopping time.

\end{document}
