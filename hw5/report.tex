\documentclass[10pt]{article}
\usepackage{latexsym}
\usepackage{amsmath}
\usepackage{natbib}
\usepackage{graphicx}
\usepackage{listings}
\usepackage{amssymb} 

\renewcommand{\P}{\mathbb{P}}

\title{MCMT Homework 5}
\author{Shun Zhang}
\date{}

\begin{document}
\maketitle

\section*{Exercise 5.1}

From Exercise 3.1, we know that $\pi$ is fully supported, i.e., $\pi(x) > 0$
for all $x$.

Assume that $\delta_x P^t$ converges to $\pi$ for some $t > 1$. Then $\delta_x
P^t = \delta_x P^{t+1} = \pi$ by the definition of stationary distribution.

Because $\pi$ is fully supported, $\pi(x) > 0$. So $\delta_x P^t (x) = \delta_x
P^{t+1} (x) > 0$. This means that $P^t(x, x) > 0$ and $P^{t+1}(x, x) > 0$. So
$t, t+1 \in T(x)$. But $gcd(t, t+1) = 1$. So $T(x) = 1$. This controdicts with
the fact that $T(x) > 1$.

\section*{Exercise 5.2}

For arbitrary $x, y \in \tilde{\Omega}$, by its definition, $\tilde{P}^t(x, y) > 0$
for some $t \geq 1$. So $\tilde{P}$ is irreducible.

For arbitrary $x \in \tilde{\Omega}$, we know $x \in \Omega$. Because $P$ has
the period of $d$, $P^d(x, x) > 0$ by definition. Because $\tilde{P} = P^d$,
$\tilde{P}(x, x) > 0$. So $1 \in T(x)$ in terms of $\tilde{P}$. So $\tilde{P}$
has the period of 1, and is aperiodic.

\end{document}
