\documentclass[10pt]{article}
\usepackage{latexsym}
\usepackage{amsmath}
\usepackage{natbib}
\usepackage{graphicx}
\usepackage{listings}
\usepackage{amssymb} 

\renewcommand{\P}{\mathbb{P}}
\newcommand{\Cov}{\mathrm{Cov}}
\newcommand{\E}{\mathrm{E}}
\newcommand{\Var}{\mathrm{Var}}

\title{MCMT Homework 11}
\author{Shun Zhang}
\date{}

\begin{document}
\maketitle

\section*{Exercise 11.1}

It is sufficient to show $\tau$ is an on-to mapping to show it is bijection. For
$x \in S_n$, $\tau^{-1} x$ exists. So $P(\tau \sigma) = P(\sigma)$. If
$\sigma$ follows uniform distribution, so does $\tau \sigma$.

\section*{Exercise 11.2}

Suppose there are $n$ cards in total and $m$ cards in the left hand. Consider
the next card to be dropped from the bottom of left-right deck or the right-hand
deck.

Following the first model, the number of cases that the next card is from the
left hand is ${n \choose m} - {n-1 \choose m} = {n-1 \choose m-1}$.
The number of cases that the next card is from the right hand: ${n-1 \choose
m}$.
The ratio of probability of the next card being from the left hand versus the
right hand is ${n-1 \choose m-1} / {n-1 \choose m} = m / (n - m)$.

Following the second model, the ratio of probability of the next card being from
the left hand versus the right hand is $\frac{m}{n} / \frac{n-m}{n} = m / (n -
m)$.

So they have the same probability distribution on the next card to be dropped.
This is applied to each step. So these two characterizations are equivalent.

\end{document}
